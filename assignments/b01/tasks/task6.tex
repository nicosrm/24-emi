% !TeX root = ../b01.tex

\section{Aufgabe BI-6}

\begin{task}
    Unter 10 Studierenden wurde ein Wettlauf veranstaltet. Die folgende Tabelle enthält die Körpergröße (Merkmal $X$) und die Platzierung (Merkmal $Y$) der 10 Teilnehmer:

    \begin{table}[H]
    \centering
    \begin{tabular}{c|cccccccccc}
        Körpergröße (in cm) & 181 & 171 & 166 & 175 & 183 &191 & 170 & 179 & 185 & 190 \\ \hline
        Platzierung & 3 & 7 & 10 & 8 & 5 & 2 & 9 & 6 & 1 & 4
    \end{tabular}
    \end{table}

    \begin{enumerate}
        \item[(a)] Ermitteln Sie Schätzwerte $\hat{a}$ und $\hat{b}$ für die Parameter $a$ und $b$ aus dem Regressionsmodell $Y=a+bX$ mit
        $a,b\in\mathbb{R}$
        nach der Methode der kleinsten Quadrate.
    \end{enumerate}
\end{task}

{\allowdisplaybreaks
    \begin{align*}
        n &= 10 \\*
        \overline{x} &= \frac{1}{10} (181 + 171 + 166 + 175 + 183 +191 + 170 + 179 + 185 + 190) = 179.1 \\*
        \overline{y} &= \frac{1}{10} (3 + 7 + 10 + 8 + 5 + 2 + 9 + 6 + 1 + 4) = 5.5 \\*
        \nonumber \\
        %
        \tilde{s}_{X,Y} &= \frac1n \sum_{j=1}^{n} (x_j - \overline{x})(y_j - \overline{y}) \\*
        &= \frac{1}{10} \big( (181 - 179.1) (3 - 5.5) + (171 - 179.1) (7 - 5.5) + (166 - 179.1) (10 - 5.5) \\*
            &\qquad\quad + (175 - 179.1) (8 - 5.5)  + (183 - 179.1) (5 - 5.5) + (191 - 179.1) (2 - 5.5) \\*
            &\qquad\quad + (170 - 179.1) (9 - 5.5) + (179 - 179.1) (6 - 5.5) + (185 - 179.1) (1 - 5.5) \\*
            &\qquad\quad + (190 - 179.1) (4 - 5.5) \big) \\*
        \tilde{s}_{X,Y} &= -20.45 \\
        \nonumber \\
        %
        \tilde{s}_X^2 &= \frac1n \sum_{j=1}^{n} (x_j - \overline{x})^2 \\*
        &= \frac{1}{10} \big( (181 - 179.1)^2 + (171 - 179.1)^2 + (166 - 179.1)^2 + (175 - 179.1)^2 \\*
            &\qquad\quad + (183 - 179.1)^2 + (191 - 179.1)^2 + (170 - 179.1)^2 + (179 - 179.1)^2 \\*
            &\qquad\quad + (185 - 179.1)^2 + (190 - 179.1)^2 \big) \\*
        \tilde{s}_X^2 &= 65.09 ~{\text{\footnotesize(positiv)}} \\
        \nonumber \\
        %
        \hat\beta &= \frac{\tilde{s}_{X,Y}}{\tilde{s}_X^2}
            = \frac{-20.45}{65.09} \approx -0.3142 \\*
        % 
        \hat\alpha &= \overline{y} - \hat\beta\cdot\overline{x} \approx 5.5 + 0.3142 \cdot 179.1 \approx 61.7697 \\*
        %
        \Rightarrow Y &= \hat\alpha + \hat\beta \cdot \overline{x} \approx 61.7697 - 0.3142x
    \end{align*}
}

Insgesamt ergibt sich die geschätzte Regressionsgerade $Y \approx 61.7697 - 0.3142x$ nach der Methode der kleinsten Quadrate.


\begin{task}
    \begin{enumerate}
        \item[(b)] Berechnen Sie das Bestimmtheitsmaß $R^2$ für das lineare Regressionsmodell gemäß (a) sowie den empirischen Korrelationskoeffizient $r_{X,Y}$ und interpretieren Sie kurz Ihre Ergebnisse.
    \end{enumerate}
\end{task}

\begin{table}[H]
\centering
\begin{tabular}{c|cc|cc}
    $x$ & $y$ & $\hat{y}$ & $(\hat{y} - \overline{y})^2$ & $(y-\overline{y})^2$ \\\hline
    181 &  3  & 4.8995    & 0.3606                       & 6.25                 \\
    171 &  7  & 8.0415    & 6.4592                       & 2.25                 \\
    166 & 10  & 9.6125    & 16.9127                      & 20.25                \\
    175 &  8  & 6.7847    & 1.6505                       & 6.25                 \\
    183 &  5  & 4.2711    & 1.5102                       & 0.25                 \\
    191 &  2  & 1.7575    & 14.0063                      & 12.25                \\
    170 &  9  & 8.3557    & 8.155                        & 12.25                \\
    179 &  6  & 5.5279    & 0.0008                       & 0.25                 \\
    185 &  1  & 3.6427    & 3.4496                       & 20.25                \\
    190 &  4  & 2.0717    & 11.7532                      & 2.25                 \\\hline
        &     &           & $\Sigma\approx64.2580$       & $\Sigma=82.5 $
\end{tabular}
\end{table}

{\allowdisplaybreaks
    \begin{align*}
        \SQE &= \sum_{j=1}^{n} (\hat{y}_j - \overline{y})^2 \approx 64.2580 \\*
        \SQT &= \sum_{j=1}^{n} (y_j - \overline{y})^2 = 82.5 \\*
        R^2 &= \frac{\SQE}{\SQT} = \frac{64.2580}{82.5} \approx 0.7789 \\*
        % r_{X,Y}^2 &= R^2 \Rightarrow r_{X,Y} = \pm\sqrt{R^2} \approx \pm\sqrt{0.7789} \approx \pm0.8825 \\
        \nonumber \\
        %
        \tilde{s}_Y^2 &= \frac1n \sum_{j=1}^{n} (y_j - \overline{y})^2 = \frac{1}{10}\cdot82.5 = 8.25 \\*
        r_{X,Y} &= \frac{\tilde{s}_{X,Y}}{\tilde{s}_X \tilde{s}_Y} = \frac{\tilde{s}_{X,Y}}{\sqrt{\tilde{s}_X^2 \tilde{s}_Y^2}}
            = \frac{-20.45}{\sqrt{65.09\cdot8.25}} \approx -0.8825
    \end{align*}
}

Insgesamt ergibt sich ein Bestimmtheitsmaß von $R^2=0.7789$ sowie ein empirischer Korrelationskoeffizient von $r_{X,Y}\approx -0.8825$. Dies weißt auf eine hohe Güte des Regressionsmodells sowie eine starke gegenläufige lineare Korrelation zwischen den beiden Merkmalen hin.


\begin{task}
    \begin{enumerate}
        \item[(c)] Bestimmen Sie (zum Vergleich) den Rangkorrelationskoeffizienten $r_{X,Y}^\star$ nach Spearman (da ja im Grunde das Merkmal $Y$ nur ordinalskaliert ist) und interpretieren Sie kurz Ihr Ergebnis.
    \end{enumerate}
\end{task}

\begin{table}[H]
\centering
\begin{tabular}{c|cccccccccc}
    $i$        & 1   & 2   & 3   & 4   & 5   & 6   & 7   & 8   & 9   & 10  \\ \hline\hline

    $x_i$      & 181 & 171 & 166 & 175 & 183 & 191 & 170 & 179 & 185 & 190 \\
    $\Rg(x_i)$ & 6   & 3   & 1   & 4   & 7   & 10  & 2   & 5   & 8   & 9   \\ \hline\hline

    $y_i$      & 3   & 7   & 10  & 8   & 5   & 2   & 9   & 6   & 1   & 4   \\
    $\Rg(y_i)$ & 3   & 7   & 10  & 8   & 5   & 2   & 9   & 6   & 1   & 4
\end{tabular}
\end{table}

Hier kommen \emph{keine} Werte in den Stichproben $x_1,\ldots,x_n$ und $y_1,\ldots,y_n$ mehrfach vor. Daher kann die vereinfachte Formel zur Bestimmung des Rangkorrelationskoeffizienten $r^\star_{X,Y}$ verwendet werden.

\begin{align*}
    r_{X,Y}^\star &= 1- \frac{6}{n(n^2-1)} \sum_{j=1}^{n} \Big(\Rg(x_j)^2 - \Rg(y_j)^2\Big) \\
    &= 1 - \frac{6}{10\cdot(10^2-1)} \sum_{j=1}^{10} \Big(\Rg(x_j)^2 - \Rg(y_j)^2\Big) \\
    &= 1 - \frac{6}{990} \Big( (6-3)^2 + (3-7)^2 + (1-10)^2 + (4-8)^2 + (7-5)^2 + (10-2)^2 \\
    &\qquad\qquad\quad~ + (2-9)^2 + (5-6)^2 + (8-1)^2 + (9-4)^2 \Big) \\
    &= 1 - \frac{1}{165}\cdot 314 \\
    \Rightarrow r_{X,Y}^\star &\approx -0.9030
\end{align*}

Es ergibt sich ein Rangkorrelationskoeffizient (nach Spearman) von $r_{X,Y}^\star \approx -0.9030$. Dies weißt auf einen stark gegenläufigen monotonen Zusammenhang hin. Dies entspricht der Folgerung aus Teilaufgabe (b), aus dem ebenfalls eine stark gegenläufige lineare Korrelation gefolgert werden konnte.
