% !TeX root = ../b01.tex

\section{Aufgabe BI-6}

\begin{task}
    Unter 10 Studierenden wurde ein Wettlauf veranstaltet. Die folgende Tabelle enthält die Körpergröße (Merkmal $X$) und die Platzierung (Merkmal $Y$) der 10 Teilnehmer:

    \begin{table}[H]
    \centering
    \begin{tabular}{c|cccccccccc}
        Körpergröße (in cm) & 181 & 171 & 166 & 175 & 183 &191 & 170 & 179 & 185 & 190 \\ \hline
        Platzierung & 3 & 7 & 10 & 8 & 5 & 2 & 9 & 6 & 1 & 4
    \end{tabular}
    \end{table}

    \begin{enumerate}
        \item[(a)] Ermitteln Sie Schätzwerte $\hat{a}$ und $\hat{b}$ für die Parameter $a$ und $b$ aus dem Regressionsmodell $Y=a+bX$ mit
        $a,b\in\mathbb{R}$
        nach der Methode der kleinsten Quadrate.
    \end{enumerate}
\end{task}

\begin{task}
    \begin{enumerate}
        \item[(b)] Berechnen Sie das Bestimmtheitsmaß $R^2$ für das lineare Regressionsmodell gemäß (a) sowie den empirischen Korrelationskoeffizient $r_{X,Y}$ und interpretieren Sie kurz Ihre Ergebnisse.
    \end{enumerate}
\end{task}

\begin{task}
    \begin{enumerate}
        \item[(c)] Bestimmen Sie (zum Vergleich) den Rangkorrelationskoeffizienten $r_{X,Y}^\star$ nach Spearman (da ja im Grunde das Merkmal $Y$ nur ordinalskaliert ist) und interpretieren Sie kurz Ihr Ergebnis.
    \end{enumerate}
\end{task}

\begin{table}[H]
\centering
\begin{tabular}{c|cccccccccc}
    $i$        & 1   & 2   & 3   & 4   & 5   & 6   & 7   & 8   & 9   & 10  \\ \hline\hline

    $x_i$      & 181 & 171 & 166 & 175 & 183 & 191 & 170 & 179 & 185 & 190 \\
    $\Rg(x_i)$ & 6   & 3   & 1   & 4   & 7   & 10  & 2   & 5   & 8   & 9   \\ \hline\hline

    $y_i$      & 3   & 7   & 10  & 8   & 5   & 2   & 9   & 6   & 1   & 4   \\
    $\Rg(y_i)$ & 3   & 7   & 10  & 8   & 5   & 2   & 9   & 6   & 1   & 4
\end{tabular}
\end{table}

Hier kommen \emph{keine} Werte in den Stichproben $x_1,\ldots,x_n$ und $y_1,\ldots,y_n$ mehrfach vor. Daher kann die vereinfachte Formel zur Bestimmung des Rangkorrelationskoeffizienten $r^\star_{X,Y}$ verwendet werden.

$$
\begin{aligned}
    r_{X,Y}^\star &= 1- \frac{6}{n(n^2-1)} \sum_{j=1}^{n} \Big(\Rg(x_j)^2 - \Rg(y_j)^2\Big) \\
    &= 1 - \frac{6}{10\cdot(10^2-1)} \sum_{j=1}^{10} \Big(\Rg(x_j)^2 - \Rg(y_j)^2\Big) \\
    &= 1 - \frac{6}{990} \Big( (6-3)^2 + (3-7)^2 + (1-10)^2 + (4-8)^2 + (7-5)^2 + (10-2)^2 \\
    &\quad + (2-9)^2 + (5-6)^2 + (8-1)^2 + (9-4)^2 \Big) \\
    &= 1 - \frac{1}{165}\cdot 314 \\
    \Rightarrow r_{X,Y}^\star &\approx -0.9030
\end{aligned}
$$

Es ergibt sich ein Rangkorrelationskoeffizient (nach Spearman) von $r_{X,Y}^\star \approx -0.9030$. Dies weißt auf einen stark gegenläufigen monotonen Zusammenhang hin.

{\color{red}
    TODO: Vergleich zu (a) hinzufügen.
}
