% !TeX root = ../b01.tex

\section{Aufgabe BI-6}

\begin{task}
    Unter 10 Studierenden wurde ein Wettlauf veranstaltet. Die folgende Tabelle enthält die Körpergröße (Merkmal $X$) und die Platzierung (Merkmal $Y$) der 10 Teilnehmer:

    \begin{table}[H]
    \centering
    \begin{tabular}{c|cccccccccc}
        Körpergröße (in cm) & 181 & 171 & 166 & 175 & 183 &191 & 170 & 179 & 185 & 190 \\ \hline
        Platzierung & 3 & 7 & 10 & 8 & 5 & 2 & 9 & 6 & 1 & 4
    \end{tabular}
    \end{table}

    \begin{enumerate}
        \item[(a)] Ermitteln Sie Schätzwerte $\hat{a}$ und $\hat{b}$ für die Parameter $a$ und $b$ aus dem Regressionsmodell $Y=a+bX$ mit
        $a,b\in\mathbb{R}$
        nach der Methode der kleinsten Quadrate.
    \end{enumerate}
\end{task}

\begin{task}
    \begin{enumerate}
        \item[(b)] Berechnen Sie das Bestimmtheitsmaß $R^2$ für das lineare Regressionsmodell gemäß (a) sowie den empirischen Korrelationskoeffizient $r_{X,Y}$ und interpretieren Sie kurz Ihre Ergebnisse.
    \end{enumerate}
\end{task}

\begin{task}
    \begin{enumerate}
        \item[(c)] Bestimmen Sie (zum Vergeich) den Rangkorrelationskoeffizienten $r_{X,Y}^*$ nach Spearman (da ja im Grunde das Merkmal $Y$ nur ordinalskaliert ist) und interpretieren Sie kurz Ihr Ergebnis.
    \end{enumerate}
\end{task}
