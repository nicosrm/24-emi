% !TeX root = ../b01.tex

\section{Aufgabe BI-1}

\begin{task}
    Die Pizzeria \textsc{Morte Dolce} (des Eigentümers \textsc{M.A. Fia}) hat zwei Lokale (kurz L1 und L2 genannt), bei denen man Mittag- und Abendessen (kurz M und A) einnehmen kann, wobei es jedoch bei den Gerichten jeweils nur grob die Unterteilung zwischen Pizza, Spaghetti, Ravioli und Cannelloni gibt. Im letzten Monat (September) wurden in jener Pizzeria wie folgt Gericht bestellt, aufgetischt und verspeist:

    \begin{table}[H]
    \centering
    \begin{tabular}{l||cc|cc||c}
        \multirow{2}{*}{} & \multicolumn{2}{c|}{\bf{L1}}  & \multicolumn{2}{c||}{\bf{L2}} &                         \\
                          & \multicolumn{1}{c}{M}         & A    & \multicolumn{1}{c}{M}  & A           & insgesamt \\ \hline\hline
        Pizza             & \multicolumn{1}{c}{400}       & 600  & 600                    & 800         & 2400      \\
        Sonstige          & \multicolumn{1}{c}{700}       & 1100 & 400                    & 400         & 2600      \\ \hline
        Summe             & \multicolumn{1}{c}{1100}      & 1700 & 1000                   & 1200        & 5000     
    \end{tabular}
    \end{table}

    \begin{enumerate}
        \item[(a)] Wie viele Merkmale werden in dieser Tabelle dargestellt, wie heißen diese und welche Merkmalsausprägungen werden hierbei jeweils berücksichtigt?
    \end{enumerate}
\end{task}

In der Tabelle werden die folgenden drei Merkmale (inkl. Merkmalsausprägung) dargestellt:

\begin{table}[H]
\centering
\begin{tabular}{l|l}
    Merkmal & Merkmalsausprägungen \\ \hline
    Lokal & L1, L2 \\
    Essenszeit & Mittagessen (M), Abendessen (A) \\
    Gericht & Pizza, Sonstige
\end{tabular}
\end{table}


\begin{task}
    \begin{enumerate}
        \item[(b)] Was (Grundgesamtheit, statistische Einheit, Merkmal usw.) stellt im Falle dieser Datenerhebung jeweils das folgende dar:
        \begin{enumerate}
            \item[($b_1$)] die Angabe L2?
            \item[($b_2$)] Herr \textsc{M. Angione}, der am 1. September mittags in L1 Cannelloni gegessen hat?
            \item[($b_3$)] die Zahl 5000?
            \item[($b_4$)] die 2400 Leute, denen eine Pizza aufgetischt wurde?
        \end{enumerate}
    \end{enumerate}
\end{task}

\begin{enumerate}
    \item[($b_1$)] Merkmalsausprägung (bzw. Beobachtungswert)
    \item[($b_2$)] Merkmalsträger (bzw. statistische Einheit)
    \item[($b_3$)] Stichprobenumfang
    \item[($b_4$)] Teilgesamtheit
\end{enumerate}
