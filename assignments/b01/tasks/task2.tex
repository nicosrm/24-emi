% !TeX root = ../b01.tex

\section{Aufgabe BI-2}

\begin{task}
    Für die Bevölkerung gewisser Regionen der Erde wurden folgende Geschlechterverteilungen (das heißt Männer 100 Frauen):
    \begin{tightcenter}
        96  101  98  96  101  98  105  106  101  104  88  97 \\
        100  96  101  92  98  104  102  97  98  93  100  94
    \end{tightcenter}
    durch statistische Erhebungen erhalten.

    \begin{enumerate}
        \item[(a)] Stellen Sie die Daten in einem Histogramm mit 5 Klassen gleicher Klassenbreite dar.
    \end{enumerate}
\end{task}

\begin{task}
    \begin{enumerate}
        \item[(b)] Berechnen Sie die Werte der (gewöhnlichen, nicht klassierten) empirischen Verteilungsfunktion $F$ der Daten an den Stellen $x_1=95.5$ und $x_2=100$.
    \end{enumerate}
\end{task}

\begin{task}
    \begin{enumerate}
        \item[(c)] Erstellen Sie den zu den Beobachtungswerten gehörigen klassischen Box-Plot (mit Kennzeichnung von Ausreißern und Extremwerten, so vorhanden).
    \end{enumerate}
\end{task}
