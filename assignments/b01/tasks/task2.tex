% !TeX root = ../b01.tex

\section{Aufgabe BI-2}

\begin{task}
    Für die Bevölkerung gewisser Regionen der Erde wurden folgende Geschlechterverteilungen (das heißt Männer 100 Frauen):
    \begin{tightcenter}
        96  101  98  96  101  98  105  106  101  104  88  97 \\
        100  96  101  92  98  104  102  97  98  93  100  94
    \end{tightcenter}
    durch statistische Erhebungen erhalten.

    \begin{enumerate}
        \item[(a)] Stellen Sie die Daten in einem Histogramm mit 5 Klassen gleicher Klassenbreite dar.
    \end{enumerate}
\end{task}

geordnete Urliste: 88 92 93 94 96 96 96 97 97 98 98 98 98 100 100 101 101 101 101 102 104 104 105 106

Stichprobenumfang $n=24$

$\sqrt{n} = \sqrt{24} \approx 4.8990 \Rightarrow \ell=5$ Klassen (nach Faustregel)

Einteilung der Klassen mit Breite $d_j=5$ wie folgt:

\begin{table}[H]
\centering
\begin{tabular}{c|ccccc}
    $j$                       & 1                   & 2         & 3          & 4                  & 5                   \\ \hline
    Klasse                    & $[85,90)$           & $[90,95)$ & $[95,100)$ & $[100,105)$        & $[105,110)$         \\
    absolute Häufigkeit $h_j$ & 1                   & 3         & 9          & 10                 & 1                   \\
    relative Häufigkeit $r_j$ & $0.041\overline{6}$ & 0.125     & 0.375      & $0.41\overline{6}$ & $0.041\overline{6}$
\end{tabular}
\end{table}

% TODO: Histogramm


\begin{task}
    \begin{enumerate}
        \item[(b)] Berechnen Sie die Werte der (gewöhnlichen, nicht klassierten) empirischen Verteilungsfunktion $F$ der Daten an den Stellen $x_1=95.5$ und $x_2=100$.
    \end{enumerate}
\end{task}

% ACHTUNG: **NICHT** klassiert!!

\begin{table}[H]
\centering
\begin{tabular}{c|llll}
     $j$      & Klasse   & $a_j$     & $r_j$               & $F(x_j)$ \\ \hline
     1        & 88       & 1         & $0.041\overline{6}$ & $0.041\overline{6}$ \\
     2        & 92       & 1         & $0.041\overline{6}$ & $0.08\overline{3}$  \\
     3        & 93       & 1         & $0.041\overline{6}$ & $0.125$             \\
     4        & 94       & 1         & $0.041\overline{6}$ & $0.1\overline{6}$   \\
     5        & 96       & 3         & $0.125$             & $0.291\overline{6}$ \\
     6        & 97       & 2         & $0.08\overline{3}$  & $0.375$             \\
     7        & 98       & 4         & $0.1\overline{6}$   & $0.541\overline{6}$ \\
     8        & 100      & 2         & $0.08\overline{3}$  & $0.625$             \\
     9        & 101      & 4         & $0.1\overline{6}$   & $0.791\overline{6}$ \\
     $\vdots$ & $\vdots$ & $\vdots$  & $\vdots$            & $\vdots$
\end{tabular}
\end{table}

$$
F(x) = \begin{cases}
    0      & \text{für}~x<88 \\
    % 0.0417 & \text{für}~88\le x<92 \\
    % 0.0833 & \text{für}~92\le x<93 \\
    % 0.125  & \text{für}~93\le x<94 \\
    \vdots & \\
    0.1\overline{6} & \text{für}~94\le x<96 \\
    \vdots & \\
    % 0.2917 & \text{für}~96\le x<97 \\
    % 0.375  & \text{für}~97\le x<98 \\
    % 0.5417 & \text{für}~98\le x<100 \\
    0.625  & \text{für}~100\le x<101 \\
    \vdots & \\
    1      & \text{für}~x\le 106
\end{cases}
$$

$\Rightarrow F(x_1) = F(95.5) = 0.1\overline{6},~~F(x_2) = F(100) = 0.625$


\begin{task}
    \begin{enumerate}
        \item[(c)] Erstellen Sie den zu den Beobachtungswerten gehörigen klassischen Box-Plot (mit Kennzeichnung von Ausreißern und Extremwerten, so vorhanden).
    \end{enumerate}
\end{task}
