% !TeX root = ../b01.tex

\section{Aufgabe BI-4}

\begin{task}
    In der Verwaltung einer bestimmten Hochschule sind 400 Personen beschäftigt. Jede Person ist entweder Arbeiter/in, angestellt oder beamtet. Die Aufteilung in Abhängigkeit vom Geschlecht ist in der Kontingenztabelle

    \begin{table}[H]
    \centering
    \begin{tabular}{l|c|c|c}
                 & Arbeiter/in & angestellt & beamtet \\ \hline
        weiblich & 5           & 160        & 42      \\
        männlich & 36          & 122        & 35     
    \end{tabular}
    \end{table}

    zusammengestellt. Bestimmen Sie für jene Daten das Kontingenzmaß $V$ nach Cramér und interpretieren Sie jenen Werte.
\end{task}

$$
\begin{aligned}
    \chi^2 &= \sum_{j=1}^m \sum_{k=1}^\ell \frac{(h_{jk} - \frac1n h_{j\bullet} h_{\bullet k})^2}{\frac1n h_{j\bullet} h_{\bullet k}} \\
    &= \frac{(5\cdot 122 - 160 \cdot 36)^2}{(5+36)(160+122)(5+160)(36+122)}
    = \frac{26522500}{301421340} \\
    \Rightarrow \chi^2 &\approx 0.0880
\end{aligned}
$$

$$
\begin{aligned}
    V &= \sqrt{\frac{\chi^2}{n (\min\lbrace m,\ell \rbrace - 1)}} \\
    &= \frac{\chi^2}{35 \cdot (\min\lbrace2,2\rbrace - 1)}
    \approx \frac{0.0880}{35 \cdot 1} = \frac{0.0880}{35} \\
    \Rightarrow V &\approx 0.0025
\end{aligned}
$$

Es ergibt sich ein Kontingenzmaß nach Cramér von $V\approx0.0025$. Dies liegt bei Nahe 0, was für einen sehr geringen Zusammenhang zwischen beiden Merkmalen spricht.
