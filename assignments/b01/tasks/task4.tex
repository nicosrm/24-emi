% !TeX root = ../b01.tex

\section{Aufgabe BI-4}

\begin{task}
    In der Verwaltung einer bestimmten Hochschule sind 400 Personen beschäftigt. Jede Person ist entweder Arbeiter/in, angestellt oder beamtet. Die Aufteilung in Abhängigkeit vom Geschlecht ist in der Kontingenztabelle

    \begin{table}[H]
    \centering
    \begin{tabular}{l|c|c|c}
                 & Arbeiter/in & angestellt & beamtet \\ \hline
        weiblich & 5           & 160        & 42      \\
        männlich & 36          & 122        & 35     
    \end{tabular}
    \end{table}

    zusammengestellt. Bestimmen Sie für jene Daten das Kontingenzmaß $V$ nach Cramér und interpretieren Sie jenen Werte.
\end{task}

Vervollständigen zur folgenden Kontingenztabelle:

\begin{table}[H]
\centering
\begin{tabular}{l|ccc|l}
             & Arbeiter/in & angestellt & beamtet & gesamt \\ \hline
    weiblich & 5           & 160        & 42      & 207    \\
    männlich & 36          & 122        & 35      & 193    \\ \hline
    gesamt   & 41          & 282        & 77      & 400
\end{tabular}
\end{table}

Daraus Bestimmung von Zwischenschritten $\frac1n h_{j\bullet} h_{\bullet k}$

\begin{table}[H]
\centering
\begin{tabular}{l|c|c|c}
    $\frac1n h_{j\bullet} h_{\bullet k}$ & Arbeiter/in                        & angestellt                          & beamtet                            \\ \hline
    weiblich                             & $\frac{41\cdot207}{400} = 21.2175$ & $\frac{282\cdot207}{400} = 145.935$ & $\frac{77\cdot207}{400} = 39.8475$ \\
    männlich                             & $\frac{41\cdot193}{400} = 19.7825$ & $\frac{282\cdot193}{400} = 136.065$ & $\frac{77\cdot193}{400} = 37.1525$
\end{tabular}
\end{table}

...sowie des $\chi^2$-Koeffizienten

$$
\begin{aligned}
    \chi^2 &= \sum_{j=1}^m \sum_{k=1}^\ell \frac{(h_{jk} - \frac1n h_{j\bullet} h_{\bullet k})^2}{\frac1n h_{j\bullet} h_{\bullet k}} \\
    &= \frac{(5-21.2175)^2}{21.2175} + \frac{(160-145.935)^2}{145.935} + \frac{(42-39.8475)^2}{39.8475} \\
    & \quad~ + \frac{(36-19.7825)^2}{19.7825} + \frac{(122-136.065)^2}{136.065} + \frac{(35-37.1525)^2}{37.1525} \\
    \Rightarrow \chi^2 &\approx 28.7412
\end{aligned}
$$

Abschließende Bestimmung des Kontingenzmaßes nach Cramér $V$:

$$
\begin{aligned}
    V &= \sqrt{\frac{\chi^2}{n (\min\lbrace m,\ell \rbrace - 1)}} \\
    &\approx \sqrt{\frac{28.7412}{400 \cdot (\min\lbrace2,3\rbrace - 1)} = \frac{28.7412}{400 \cdot 1}} \\
    \Rightarrow V &\approx 0.2681
\end{aligned}
$$

Es ergibt sich ein Kontingenzmaß nach Cramér von $V\approx0.2581$. Somit gilt $0.2\le V<0.6$, was für einen mittleren Zusammenhang zwischen beiden Merkmalen spricht.
