% !TeX root = ../b01.tex

\section{Aufgabe BI-5}

\begin{task}
    Eine Gesamtstichprobe von 20 Elementen, bei der man sich für zwei Merkmale $X$ und $Y$ interessiert, wurde zur Datenerhebung in zwei gleich große Teilstichproben aufgeteilt. Für die Teilstichproben ergaben sich dabei die folgenden Maßzahlen:

    \begin{table}[H]
    \centering
    \begin{tabular}{c|cccccc}
        Teilstichprobe $j$ & $n_j$ & $\overline{x}_{n_j}^{(j)}$ & $\overline{y}_{n_j}^{(j)}$ & $\tilde{s}_{X,X}^{(j)}$ & $\tilde{s}_{Y,Y}^{(j)}$ & $r_{X,Y}^{(j)}$ \\ \hline
        1 & 10 & 12 & 0 & 36 & 9  & 1 \\
        2 & 10 & 0  & 6 & 9  & 36 & 1
    \end{tabular}
    \end{table}

    Wie groß ist dann der Korrelationskoeffizient nach Bravais-Pearson $r_{X,Y}$ für die Gesamtstichprobe?
\end{task}
